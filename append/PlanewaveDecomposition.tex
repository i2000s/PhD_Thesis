\chapter{Cylindrical function decomposition of a tilted incident plane wave}\label{Ch:PlanewaveDecomposition}
As an example of applying the mode decomposition technique we used above to find the projected bounded and unbounded modes under cylindrical boundary conditions, below, we demonstrate a simple case on decomposing a tilted incident plane wave into corresponding cylindrical modes. The solution might be useful to give us some insights on solving the corresponding nanofiber problem when an external field presents, which is the case for some cooling and state preparation protocols demonstrated in experiments~\cite{Meng2017ground,Ostfeldt2017Dipole}. 

The key to solve this kind of problems is to decompose all field functions into cylindrical functions. The bare nanofiber modes have already been decomposed into cylindrical functions in the last section; therefore, here, we only need to decompose the incident field as cylindrical functions. 

We assume the incident plane wave is given by
\begin{equation}
\mathbf{E}(\br,t)=\re\left[\mathbf{U}(\br,t) \right]=\mathbf{E}_0 \cos (\mathbf{k}\cdot\mathbf{r}-\omega t + \phi_0),
\end{equation}
where the forward propagating wave can be given by
\begin{align}
\mathbf{U}(\br,t) &= \mathbf{U}_0e^{i(\mathbf{k}\cdot\mathbf{r}-\omega t + \phi_0)}\\
&= \mathbf{U}_0e^{i\mathbf{k}\cdot\mathbf{r}}e^{i(\phi_0-\omega t )}.
\end{align}
with the initial phase, $\phi_0$, and the vector amplitude of $\mathbf{U}_0$. We can ignore the phase offset, and separate the spatial and temporal parts for the forward-propagating field. We want to expand the plane wave function into cylindrical functions, and thus we can apply the technique we used in the last appendix to solve the boundary condition problem and decompose the bound and radiation modes. The only term that needs to be expanded is the $ e^{i\mathbf{k}\cdot\mathbf{r}} $ factor. 

We define $ \mathbf{k}\cdot\mathbf{r}=(k\!_{\perp}\mathbf{e}\!_{k\!_\perp}+k_z\mathbf{e}_{z}) \cdot(r\!_{\perp}\mathbf{e}\!_{r\!_\perp}+z\mathbf{e}_{z}) = k\!_{\perp}r\!_{\perp}\cos(\phi_{k}-\phi_{r})+k_{z}{z}= k\!_\perp r\!_\perp \cos \Delta\phi +k_{z}{z}$, where $ \Delta\phi=\phi_{k}-\phi_{r} \in [0,2\pi)$ is the angle between $ \mathbf{e}_{k\!_\perp} $ and $ \mathbf{e}_{r\!_\perp} $. Thus
\begin{align}
e^{i\mathbf{k}\cdot \mathbf{r}}=e^{ik\!_\perp r\!_\perp\cos\Delta\phi}e^{ik_{z}{z}}
\end{align}
is a periodic function of $ \Delta\phi $ and hence can be expanded into a Fourier series given below. 
\begin{align}
e^{i\mathbf{k}\cdot \mathbf{r}} &= \sum_{m=-\infty}^{\infty}c_{m}(k\!_\perp r\!_\perp)e^{im\Delta\phi}e^{ik_{z}{z}},
\end{align}
where the coefficients $ c_{m}(k\!_\perp r\!_\perp) $ is associated with Bessel's first integral\index{Bessel function!Bessel's first integral}~\footnote{see Jackson's E\&M of Ref.~\cite{Jackson1975}, on page 140.}
\begin{align}
c_{m}(k\!_\perp r\!_\perp)&= \frac{1}{2\pi} \int_0^{2\pi} e^{ik\!_\perp r\!_\perp\cos \Delta\phi}e^{-im\Delta\phi}\mathrm{d}\Delta\phi\\
&=i^mJ_m(k\!_\perp r\!_\perp).
\end{align}
Therefore, we have
\begin{align}
e^{i\mathbf{k}\cdot \mathbf{r}} &=\sum_{m=-\infty}^{\infty}i^me^{im\Delta\phi}e^{ik_{z}{z}}J_m(k\!_\perp r\!_\perp).
\end{align}

Notice that the first kind of Bessel's function\index{Bessel function!Bessel function of the first kind} usually represent standing radial waves, while Hankel functions\index{Hankel function} usually describe traveling waves. Using the relationships that $ J_{m}(k\!_\perp r\!_\perp)=H_{m}^{(1)}(k\!_\perp r\!_\perp)+H_{m}^{(2)}(k\!_\perp r\!_\perp) $, we can re-express the plane wave in terms of incoming (at negative r) and outgoing (at positive r) as below.
\begin{align}
e^{i\mathbf{k}\cdot \mathbf{r}} &=\sum_{m=-\infty}^{\infty}i^me^{im\Delta\phi}e^{ik_{z}{z}} H_{m}^{(1)}(k\!_\perp r\!_\perp)+\sum_{m=-\infty}^{\infty}i^me^{im\Delta\phi}e^{ik_{z}{z}} H_{m}^{(2)}(k\!_\perp r\!_\perp).
\end{align}

This result shows that only when $ k_z=\beta $ can the tilted plane wave contributes to nanofiber modes be with wavenumber $ \beta $, since both $ k_z $ and $ \beta $ have consistent physics meaning. Therefore, when a plane wave comes perpendicular to the fiber axis, the wave can rarely couple to the fiber's guided modes, which is good for minimizing the influence of the incident external field directly mixed into the detected signal at the end of the fiber. 