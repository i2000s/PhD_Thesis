\chapter{Maxwell's Equations for an waveguide}
This chapter will dedicate to some fundamental theory of Maxwell's equations. More detailed 
discussions can be found in Ref.~\cite{Snyder1983Optical}. Some detailed solution of Maxwell's equation 
applied to cylindrical structures can be found in Ref.~\cite{Wait1959}. 

For non-magnetic materials which normally constitute an optical waveguide with $ \mu=\mu_0 $, the 
spatial dependence of the electrical field $ \boldsymbol{\mathcal{E}}(\br) $ and the magnetic field $ 
\boldsymbol{\mathcal{H}}(\br) $ of an optical waveguide is determined by Maxwell's equations:
\begin{align}
\nabla\times \boldsymbol{\mathcal{E}} &= i(\mu_0/\varepsilon_0)^{1/2} k \boldsymbol{\mathcal{H}}, & 
\nabla\times \boldsymbol{\mathcal{H}} &= \boldsymbol{\mathcal{J}}-i(\varepsilon_0/\mu_0)^{1/2}kn^2 
\boldsymbol{\mathcal{E}}, \label{EHtimesMKS}\\
\nabla\cdot (n^2 \boldsymbol{\mathcal{E}}) &= \rho/\varepsilon_0, & \nabla\cdot 
\boldsymbol{\mathcal{H}} &=0, \label{EHdotsMKS}
\end{align}
where $ \boldsymbol{\mathcal{J}} $ and $ \rho $ are the current density and charge density, $ 
\varepsilon=n^2 \varepsilon_0 $ is the dielectric constant of the waveguide as a function of space, and $ 
k_0=2\pi/\lambda=\omega_0/c $.  We usually assume an implicit chromatic time dependence factor $ \exp(-i\omega_0 t) $ in the 
full field vectors. These equations above are written in MKS units. 

Sometimes, to avoid the dielectric constant $ \varepsilon_0 $ and magnetic permeability $ \mu_0 $ of 
free space, we can rewrite the Maxwell's equations in Gaussian-cgs units, which yield
\begin{align}
\nabla \times \boldsymbol{\mathcal{E}} &= -\frac{1}{c} \pt{\boldsymbol{\mathcal{B}}}=i 
k_0\boldsymbol{\mathcal{B}}, & \nabla \times 
\boldsymbol{\mathcal{H}} &= \frac{4\pi}{c} \boldsymbol{\mathcal{J}} + 
\frac{1}{c}\pt{\boldsymbol{\mathcal{D}}}= \frac{4\pi}{c}\boldsymbol{\mathcal{J}} 
-ik_0\boldsymbol{\mathcal{D}}, \label{eq:maxwelldiv}\\
\nabla \cdot \boldsymbol{\mathcal{D}} &= 4\pi \rho, & \nabla \cdot \boldsymbol{\mathcal{B}} &=0,\label{eq:maxwellgrad}
\end{align}
where $ \boldsymbol{\mathcal{D}} = \varepsilon \boldsymbol{\mathcal{E}}=\varepsilon_r \boldsymbol{\mathcal{E}} $ and $ 
\boldsymbol{\mathcal{B}}= \mu\boldsymbol{\mathcal{H}}\approx \boldsymbol{\mathcal{H}} $ for non-magnetic materials. By default, we will use Gauss units in this piece of work, and only consider non-magnetic materials. Notice that, we make the dielectric constant equal to the relative dielectric constant ($ \varepsilon=\varepsilon $) in the Gaussian-cgs units convention.  

Equations~\eqref{eq:maxwelldiv} can lead to the wave equation for a chromatic electric field in Gauss units as
\begin{align}
-\nabla \times (\nabla \times \boldsymbol{\mathcal{E}})+\varepsilon k_0^2\boldsymbol{\mathcal{E}}=-ik_0\frac{4\pi}{c} \boldsymbol{\mathcal{J}}. \label{eq:waveeqGaussU}
\end{align}
Similarly, the corresponding wave equation in SI units can be given by
\begin{align}
-\nabla \times (\nabla \times \boldsymbol{\mathcal{E}})+ n^2 k_0^2\boldsymbol{\mathcal{E}}=-ik_0\sqrt{\frac{\mu_0}{\varepsilon_0}} \boldsymbol{\mathcal{J}}.
\end{align}

We can further simplify the equation above by using 
\begin{align}
\nabla\times (\nabla\times \bmc{E}) &= -\nabla(\nabla\cdot \bmc{E})+ \nabla^2\bmc{E}. 
\end{align}
In the case that there is no net charge and current in the space, the wave equations above in both Gauss and SI units can be simplified to be
\begin{align}\label{eq:freespacewaveeq}
(\nabla^2 + n^2k_0^2) \boldsymbol{\mathcal{E}}=0.
\end{align}

\section{Fields of translationally invariant waveguides}
We define the axis of the waveguide is along $ z $-direction, and the refractive index profile of the 
waveguide is independent of $ z $, i.e. $ n=n(\br\!_\perp) $, which means the waveguide is 
translationally invariant. The fields of the waveguide can then be rewritten in a separable form
\begin{align}
\bmc{E}(\br) &= \bmc{E}(\br\!_\perp)\exp(i\beta z), & \bmc{H}(\br) &= \bmc{H}(\br\!_\perp)\exp(i\beta 
z),
\end{align}
where $ \beta $ is the propagation constant and $ \br\!_\perp $ is the position vector in the transverse 
plane perpendicular to the $ z $-axis. We can further decompose these fields into longitudinal and 
transverse components, parallel and orthogonal to the waveguide axis, respectively, and denoted by 
subscripts $ z $ and $ \perp $. That is
\begin{align}\label{EHtranslong}
\bmc{E}(\br) &= \left( \bmc{E}\!_\perp + \mathcal{E}_z \hat{e}_z \right)\exp(i\beta z), & \bmc{H}(\br) &= 
\left( \bmc{H}\!_\perp + \mathcal{H}_z \hat{e}_z\right) \exp(i\beta z)
\end{align}

Notice that, in literature, many authors use the sign parameter $ f=\pm 1 $ in front of $ \beta $ to indicate the propagating direction of the fields. Here, we define forward- and backward-propagating waves using the sign of the product of $ \beta z $. This is convenient to differ the contour integration paths when we decompose the radiation and guided modes for $ z=z-z'>0 $ or $ z=z-z'<0 $ cases. 

\section{Relationships between field components}\label{MWE:components}
By substituting Equ.~\ref{EHtranslong} into the source-free Maxwell equations 
(Equs.~\ref{EHtimesMKS} and~\ref{EHdotsMKS} with $ \bmc{J}=0,\, \rho=0 $), and comparing 
longitudinal and transverse components, we obtain the relationships between fields components as 
below
\begin{subequations}
\label{EHtz0}
\begin{align}
\bmc{E}\!_\perp &= -\left(\frac{\mu_0}{\varepsilon_0} \right)^{1/2} \frac{1}{k_0n^2} \hat{e}_z \times 
\left(\beta \bmc{H}\!_\perp + i\nabla\!_\perp \mathcal{H}_z \right),\\
\bmc{H}\!_\perp &= \left(\frac{\varepsilon_0}{\mu_0} \right)^{1/2} \frac{1}{k_0} \hat{e}_z \times \left(\beta 
\bmc{E}\!_\perp +i\nabla\!_\perp \mathcal{E}_z \right),\\
\mathcal{E}_z &= i\left(\frac{\mu_0}{\varepsilon_0} \right)^{1/2} \frac{1}{k_0n^2} \hat{e}_z \cdot 
\nabla\!_\perp \times \bmc{H}\!_\perp = \frac{i}{\beta} \left( \nabla\!_\perp \cdot \bmc{E}\!_\perp + 
(\bmc{E}\!_\perp \cdot \nabla\!_\perp) \ln n^2 \right),\\
\mathcal{H}_z &= -i \left(\frac{\varepsilon_0}{\mu_0} \right)^{1/2} \frac{1}{k_0}\hat{e}_z \cdot 
\nabla\!_\perp \times \bmc{E}\!_\perp = \frac{i}{\beta} \nabla\!_\perp \cdot \bmc{H}\!_\perp. 
\end{align}
\end{subequations}

Now we consider the waveguide of cylindrical fiber case. Due to symmetry, we tend to use the 
cylindrical coordinate system, which gives 
\begin{align}
\nabla\!_\perp = \hat{r}\!_\perp \pp{}{r\!_\perp} + \hat{\phi} \frac{1}{r\!_\perp} \pp{}{\phi}.
\end{align}
Therefore, Equ.~\ref{EHtz0} implies that the transverse components can be expressed in terms of the 
longitudinal components by
\begin{subequations}
\begin{align}
\mathcal{E}_{r\!_\perp} &= \frac{i}{\kappa_i^2} \left[ \beta \pp{\mathcal{E}_z }{r\!_\perp} + 
\left(\frac{\mu_0}{\varepsilon_0} \right)^{1/2}  \frac{k_0}{r\!_\perp} 
\pp{\mathcal{H}_z}{\phi} \right], \\
\mathcal{E}_\phi &= \frac{i}{\kappa_i^2} \left[\frac{\beta}{r\!_\perp} 
\pp{\mathcal{E}_z}{\phi} -\left(\frac{\mu_0}{\varepsilon_0} \right) ^{1/2} k_0 
\pp{\mathcal{H}_z}{r\!_\perp} 
\right],\\
\mathcal{H}_{r\!_\perp} &= \frac{i}{\kappa_i^2} \left[ \beta \pp{\mathcal{H}_z}{r\!_\perp} 
-\left(\frac{\varepsilon_0}{\mu_0} \right)^{1/2} \frac{k_0n^2}{r\!_\perp}\pp{\mathcal{E}_z}{\phi} \right],\\
\mathcal{H}_\phi &= \frac{i}{\kappa_i^2} \left[ \frac{\beta}{r\!_\perp} \pp{\mathcal{H}_z}{\phi} + 
\left(\frac{\varepsilon_0}{\mu_0} \right)^{1/2} k_0n^2 \pp{\mathcal{E}_z}{r\!_\perp} \right],
\end{align}
\end{subequations}
with $ \kappa_i^2= k_0^2n_i^2-\beta^2=k_0^2 \varepsilon_i -\beta^2  $ and $ n=n(\br\!_\perp)=n_i$ for corresponding region $i$. 

Correspondingly, the component relationships in Gaussian-cgs units are
\begin{subequations}\label{EHzgauss}
\begin{align}
\mathcal{E}_{r\!_\perp} &= \frac{i}{\kappa_i^2} \left[ \beta \pp{\mathcal{E}_z }{r\!_\perp} + 
  \frac{k_0}{r\!_\perp} 
\pp{\mathcal{H}_z}{\phi} \right]= \frac{i\beta}{\kappa_i^2} \pp{\mathcal{E}_z }{r\!_\perp} - 
  \frac{k_0m}{r\!_\perp \kappa_i^2} {\mathcal{H}_z}, \\
\mathcal{E}_\phi &= \frac{i}{\kappa_i^2} \left[\frac{\beta}{r\!_\perp} 
\pp{\mathcal{E}_z}{\phi} - k_0 
\pp{\mathcal{H}_z}{r\!_\perp} 
\right] = -\frac{\beta m}{r\!_\perp \kappa_i^2} 
{\mathcal{E}_z} - \frac{ik_0}{\kappa_i^2} 
\pp{\mathcal{H}_z}{r\!_\perp},\\
\mathcal{H}_{r\!_\perp} &= \frac{i}{\kappa_i^2} \left[ \beta \pp{\mathcal{H}_z}{r\!_\perp} 
- \frac{k_0n^2}{r\!_\perp}\pp{\mathcal{E}_z}{\phi} \right]= \frac{i\beta}{\kappa_i^2} \pp{\mathcal{H}_z}{r\!_\perp} 
+ \frac{k_0n^2m}{r\!_\perp \kappa_i^2} {\mathcal{E}_z},\\
\mathcal{H}_\phi &= \frac{i}{\kappa_i^2} \left[ \frac{\beta}{r\!_\perp} \pp{\mathcal{H}_z}{\phi} + 
 k_0n^2 \pp{\mathcal{E}_z}{r\!_\perp} \right] = -\frac{\beta m}{r\!_\perp \kappa_i^2} {\mathcal{H}_z} + 
  \frac{ik_0n^2}{\kappa_i^2} \pp{\mathcal{E}_z}{r\!_\perp}.
\end{align}
\end{subequations}


%\textcolor{red}{Vector and Scalar operators in various coordinate systems...}

