\documentclass[preprint,aps,pra,onecolumn]{revtex4-1}
\usepackage{etex}
%\usepackage{amsmath}
\usepackage{bm}
\usepackage{bbm}
\usepackage{listings}
% % \textwidth 16cm \textheight 23.5cm
% \renewcommand{\baselinestretch}{1.2}
%\usepackage{graphicx}
%\usepackage{graphics}
\usepackage{epsfig}
\usepackage{color}
\usepackage{multirow}
\usepackage[colorlinks]{hyperref}
\usepackage{fancyhdr}
\usepackage{natbib} %[numbers]
\usepackage{bibentry}
% underline tool
\usepackage[normalem]{ulem}
\usepackage{xcolor}
%\uline{foo}	Underlines foo
%\uuline{foo}	Double underlines foo
%\uwave{foo}	Underlines foo with a wavy line
%\sout{foo}	Strikesout foo
%\xout{foo}	Crosses out foo with ¡®/6¤7¡¯
% triple lines in colors
\makeatletter
\newcommand\uuuline{\bgroup\markoverwith%
   {%
     \textcolor{red}{\rule[-0.5ex]{2pt}{0.4pt}}%
     \llap{\textcolor{blue}{\rule[-0.7ex]{2pt}{0.4pt}}}%
     \llap{\textcolor{green}{\rule[-0.9ex]{2pt}{0.4pt}}}%
   }%
   \ULon}
\makeatother


\usepackage{amsmath,soul} % underline with a number
% usage example: $\underset{4}{\text{\ul{This is short text}}}$
% another package to use, but did not work for me.
% From: http://tex.stackexchange.com/questions/45341/labeling-underlined-text-over-multiple-lines
%\usepackage{soulpos}
%\ulposdef{\ulnumaux}{%
%   $\underset{\saveulnum}{\rule[-.7ex]{\ulwidth}{.4pt}}$}
%
%\newcommand{\ulnum}[2]{%
%  \def\saveulnum{#1}%
%  \ulnumaux{#2}}

% todo list and commands
\usepackage{todonotes}
%% to avoid the conflict with amths package % not working
%\makeatletter
%\providecommand\@dotsep{5}
%\makeatother
%\listoftodos\relax
\usepackage{makeidx}
\allowdisplaybreaks
% for eps transfering to pdf.
\usepackage[update,prepend]{epstopdf}
\usepackage{ifpdf}

\ifpdf
   \usepackage{graphicx}
   \usepackage{epstopdf}
   \epstopdfsetup{suffix=}
   \DeclareGraphicsRule{.eps}{pdf}{.pdf}{`epstopdf #1}
   \pdfcompresslevel=9
\else
   \usepackage{graphicx}
\fi
% subfig
\usepackage{mwe}
\usepackage{subfig}
% to fix a figure's position using [H] option of thec figure.
\usepackage{float}
% to use \lesssim and other math symbols
\usepackage{amssymb}


% self-defined short-cuts and commands

% packages we need for judging the operating system
% compile your tex file with option -shell-escape is required: 
% e.g. xelatex -shell-escape file.tex
\usepackage{pdftexcmds}
\usepackage{catchfile}
\usepackage{ifluatex}
\usepackage{ifplatform}
%\input{Mydef.tex}
% judge platform and include correct definiation package
\ifwindows
	%\input{F:/Research/Works/Templates/Mydef.tex} % %
	\input{F:/Research/Works/Templates/Mydef.tex} % 
\else
	\input{/media/F/Research/Works/Templates/Mydef.tex} %
\fi


%\includeonly{VectorTensor}
% for table captions.
\usepackage{tabularx,ragged2e,booktabs} %,caption

% packages for drawing
\usepackage{tikz}
%\usepackage{xypic}
%\usepackage[matrix,frame,arrow]{xypic}
\usetikzlibrary{calc}
%\input{Qcircuit}
\usepackage{qcircuit}

% Redefine the tensor command.
\renewcommand{\tensor}[1]{\boldsymbol{#1}}





\begin{document}
%opening
\title{A short note on factors in front of dyadic Green function and decay rate formulas}
\author{Xiaodong Qi}
\date{\today}
%\maketitle

\begin{abstract}
This document summarizes some important formulas used to express dyadic Green functions and decay 
rates which have been shown fundamentally import to calculate Faraday and Birefringence effects for 
our nanofiber project. 
\end{abstract}

\section{From decay rates to dyadic Green functions}

From Le Kien's results~\cite{LeKien2005,LeKien2014}, we can summarize the decay rate formulas as 
below:
\begin{align}
\Gamma_{\mu}^{gyd} &= \frac{n_g\omega}{2\varepsilon_0 \hbar c} \sum_g \mathbf{d}_{eg}\cdot 
\mathbf{u}_{\mu}(\br^\prime\!_\perp) \mathbf{u}_{\mu}^* (\br^\prime\!_\perp) \cdot \mathbf{d}_{ge} 
\quad (SI)\\
&=  \frac{2\pi n_g\omega}{ \hbar c} \sum_g \mathbf{d}_{eg}\cdot \mathbf{u}_{fm}(\br^\prime\!_\perp) 
\mathbf{u}_{fm}^* (\br^\prime\!_\perp) \cdot \mathbf{d}_{ge} \quad (cgs) \\ 
\Gamma_{\nu }^{rad} &=  \frac{\omega}{2\varepsilon_0 \hbar c} \sum_g \int_{-n_2 k }^{n_2 
k}\mathrm{d}\beta\quad \mathbf{d}_{eg}\cdot 
\mathbf{u}_{\nu}(\br^\prime\!_\perp) \mathbf{u}_{\nu}^* (\br^\prime\!_\perp) \cdot \mathbf{d}_{ge} 
\quad (SI)\\
&=  \frac{2\pi \omega}{ \hbar c} \sum_g \int_{-n_2 k }^{n_2 k}\mathrm{d}\beta \quad 
\mathbf{d}_{eg}\cdot 
\mathbf{u}_{\nu}(\br^\prime\!_\perp) \mathbf{u}_{\nu}^* (\br^\prime\!_\perp) \cdot \mathbf{d}_{ge} 
\quad (cgs).
\end{align}

From the book, \textit{Principles of Nano-Optics}~\cite{Novotny2012} and the paper by 
Sondergaard~\cite{Sondergaard2001}, we can also summarize some of the decay rate formulas as below:
\begin{align}
\Gamma_0 &= \sum_g \frac{\omega^3 |\mathbf{d}_{eg}|^2}{3\pi\varepsilon_0\hbar c^3} \quad (SI)\\
&= \frac{4}{3} \left(\frac{\omega}{c} \right)^3 \sum_g  \frac{|\mathbf{d}_{eg}|^2}{\hbar}\quad (cgs)\\
&= - \frac{8\pi}{\hbar} \left( \frac{\omega}{c}\right)^2\sum_g \mathbf{d}_{eg} \cdot \mathrm{Im} 
\left[ \mathbf{G}^{(0)} (\br^\prime\!_\perp,\br^\prime\!_\perp)\right]\cdot \mathbf{d}_{ge} \quad (cgs)\\
\Gamma &= - \frac{2}{\varepsilon_0\hbar} \left( \frac{\omega}{c}\right)^2\sum_g \mathbf{d}_{eg} \cdot 
\mathrm{Im} \left[
\mathbf{G} (\br^\prime\!_\perp,\br^\prime\!_\perp)\right] \cdot \mathbf{d}_{ge} \quad (SI)\\
&=  - \frac{2\mu_0\omega^2}{\hbar} \sum_g  \mathbf{d}_{eg} \cdot \mathrm{Im} \left[ 
\mathbf{G} (\br^\prime\!_\perp,\br^\prime\!_\perp)\right]\cdot \mathbf{d}_{ge} \quad (SI)\\
&= - \frac{8\pi}{\hbar} \left( \frac{\omega}{c}\right)^2\sum_g \mathbf{d}_{eg} \cdot \mathrm{Im} 
\left[ \mathbf{G} (\br^\prime\!_\perp,\br^\prime\!_\perp)\right]\cdot \mathbf{d}_{ge} \quad (cgs).
\end{align}
The dyadic Green functions above are all transverse dyads for our case, and can be specified for a given 
mode contribution component.  Also, $ \Gamma_0 $ with the subscript $ 0 $ is the decay rate in free 
space. 

Comparing the two sets of decay 
rate formulas in terms of eigenmodes and dyadic Green functions, we can find that 
\begin{align}
\mathrm{Im}\left[ \mathbf{G}^{gyd}_\mu (\br^\prime\!_\perp, \br^\prime\!_\perp) \right] &= 
-\frac{n_g}{4k} \mathbf{u}_{\mu}(\br^\prime\!_\perp) \mathbf{u}_{\mu}^* 
(\br^\prime\!_\perp)\label{eq:Ggydmu} \\
\mathrm{Im}\left[ \mathbf{G}^{rad}_\mu (\br^\prime\!_\perp, \br^\prime\!_\perp) \right] &= 
-\frac{1}{4k}  \int_{-n_2 k }^{n_2 k}\mathrm{d}\beta \quad 
\mathbf{u}_{\nu}(\br^\prime\!_\perp) \mathbf{u}_{\nu}^* (\br^\prime\!_\perp)\\
\mathrm{Im}\left[ \mathbf{G}^{(0)} (\br^\prime\!_\perp, \br^\prime\!_\perp) \right] &= \mathrm{Im} 
\left[ \mathbf{G}^{rad}_{(0)}(\br'\!_\perp,\br'\!_\perp)\right]  = -\frac{ 
k}{6\pi}\eye = -\frac{ 
\omega}{6\pi c}\eye,\label{eq:G0}
\end{align}
with $ k=\omega/c $. 


Combined with Klimov's paper~\cite{Klimov2004}, we can also rewrite the decay rates in units of the
vacuum decay rate as
\begin{align}
\frac{\Gamma^{gyd}_\mu}{\Gamma_0} &= - \frac{6\pi c}{\omega}\frac{\sum_g\mathbf{d}_{eg}\cdot 
\mathrm{Im}\left[\mathbf{G}_\mu^{gyd}(\br'\!_\perp,\br'\!_\perp) \right]\cdot\mathbf{d}_{eg}^* 
}{\sum_g|\mathbf{d}_{eg}|^2} =  \frac{3}{2} \frac{\mathrm{Im}\left[\sum_g\mathbf{d}_{eg}^*\cdot 
\mathbf{E}_{(R,\mu)}^{gyd}(\br') \right]}{\sum_g|\mathbf{d}_{eg}|^2 k^3}\nonumber\\
&=\frac{3\pi}{2}\left(\frac{c}{\omega}\right)^2 n_g  \frac{\sum_{g}|\mathbf{d}_{eg}\cdot 
\mathbf{u}_\mu(\br'\!_\perp)|^2}{\sum_g|\mathbf{d}_{eg}|^2}\\
\frac{\Gamma^{rad}_\nu}{\Gamma_0} &= - \frac{6\pi c}{\omega}\frac{\sum_g\mathbf{d}_{eg}\cdot 
\mathrm{Im}\left[\mathbf{G}_\nu^{rad}(\br'\!_\perp,\br'\!_\perp) \right]\cdot\mathbf{d}_{eg}^* 
}{\sum_g|\mathbf{d}_{eg}|^2} =  1+ \frac{3}{2} \frac{\mathrm{Im}\left[\sum_g\mathbf{d}_{eg}^*\cdot 
\mathbf{E}_{(R,\nu)}^{rad}(\br'\!_\perp) \right]}{\sum_g|\mathbf{d}_{eg}|^2 k^3}\nonumber\\
&=1+\frac{3\pi}{2}\left(\frac{c}{\omega}\right)^2\int_{-n_2k}^{n_2k}\mathrm{d}\beta  
\frac{\sum_{g}|\mathbf{d}_{eg}\cdot 
\mathbf{u}_\nu(\br'\!_\perp)|^2}{\sum_g|\mathbf{d}_{eg}|^2}\\
\mathbf{E}_{(R,\mu)}^{gyd}(\br'\!_\perp) &= - \frac{1}{4\pi k^2}  
\mathbf{G}^{gyd}_{(R,\mu)}(\br'\!_\perp,\br'\!_\perp)\cdot 
\mathbf{d}_{eg} \\
\mathbf{E}_{(R,\nu)}^{rad}(\br'\!_\perp) &= - \frac{1}{4\pi k^2}  
\mathbf{G}^{rad}_{(R,\nu)}(\br'\!_\perp,\br'\!_\perp)\cdot 
\mathbf{d}_{eg}.
\end{align}
The index $ R $ indicates the component of reflection, and $\mu=(\omega_0,m,f)  $ identifies the 
normalized  fiber mode $ \mathbf{u}_\mu(\br'\!_\perp) $. I have used the radiation decomposition of the 
dyadic Green function relation that  
\begin{align}
\mathrm{Im} \left[ \mathbf{G}(\br'\!_\perp,\br'\!_\perp) \right] &= 
\mathrm{Im} \left[ \mathbf{G}^{(0)}(\br'\!_\perp,\br'\!_\perp)\right] +\mathrm{Im} \left[ 
\mathbf{G}^{(R)}(\br'\!_\perp,\br'\!_\perp)\right] \\
&=
\mathrm{Im} \left[ \mathbf{G}^{rad}_{(0)}(\br'\!_\perp,\br'\!_\perp)\right] +\mathrm{Im} \left[ 
\mathbf{G}^{gyd}_{(R,\mu)}(\br'\!_\perp,\br'\!_\perp) \right] 
+\mathrm{Im} \left[ \mathbf{G}^{rad}_{(R,\nu)}(\br'\!_\perp,\br'\!_\perp)\right]
\end{align}
for $ \br'\!_\perp\ge a $.

All the results above are consistent analytically, and have also been examined numerically.  


\section{Inspiring our paper writing work}
Here are a few differences of the results above from the paper draft we are working on:

Equ.~\eqref{eq:Ggydmu} is $ 1/2 $ of the the one given in the current paper draft, which means the phase shift calculated from dyadic Green function method will also have a factor of $ 1/2 $ off. 

However, the paper draft is correct on expressing $ \Gamma $'s in terms of eigenmodes. It could be just a business of defining dyadic Green functions. 

%(2) Equ.~\eqref{eq:G0} seems also to be $1/2$ of the previous result on our old notes, which lead to a 
%change of the 
%phase shift formula for the free-space case. 




%\bibliography
%\bibliographystyle{amsplain}
\bibliographystyle{unsrt}
%\bibliography{NanofiberArchive}
% \nocite{*}
\ifwindows
	\bibliography{F:/References/Archive/Archive}
\else
	\bibliography{/media/F/References/Archive/Archive}
\fi
\end{document}
