\chapter{Cylindrical function decomposition of a tilt incident plane wave}\label{Ch:PlanewaveDecomposition}
As an example of applying the technique we used above to solve the bound and unbound modes under cylindrical boundary conditions, below, we show an application on solving the nanofiber problem with a tilt incident plane wave. 

The key to solve this kind of problems is to decompose all field functions into cylindrical functions. The bare nanofiber modes have already been decomposed into cylindrical functions in the last section; therefore, here, we only need to decompose the incident field as cylindrical functions. 

We assume the incident plane wave is given by
\begin{equation}
\mathbf{E}(\br,t)=\Re\left[\mathbf{U}(\br,t) \right]=\mathbf{E}_0 \cos (\mathbf{k}\cdot\mathbf{r}-\omega t + \phi_0),
\end{equation}
where the forward propagating wave can be given by
\begin{align}
\mathbf{U}(\br,t) &= \mathbf{U}_0e^{i(\mathbf{k}\cdot\mathbf{r}-\omega t + \phi_0)}\\
&= \mathbf{U}_0e^{i\mathbf{k}\cdot\mathbf{r}}e^{i(\phi_0-\omega t )}.
\end{align}
with the initial phase, $\phi_0$, and the vector amplitude of $\mathbf{U}_0$. We can ignore the phase offset, and separate the spatial and temporal parts for the forward-propagating field. We want to expand the plane wave function into cylindrical functions, and thus we can apply the technique we used in the last section to solve the boundary condition problem and decompose the bound and radiation modes. The only term that needs to be expanded is the $ e^{i\mathbf{k}\cdot\mathbf{r}} $ factor. 

We define $ \mathbf{k}\cdot\mathbf{r}=(k\!_{\perp}\mathbf{e}\!_{k\!_\perp}+k_z\mathbf{e}_{z}) \cdot(r\!_{\perp}\mathbf{e}\!_{r\!_\perp}+z\mathbf{e}_{z}) = k\!_{\perp}r\!_{\perp}\cos(\phi_{k}-\phi_{r})+k_{z}{z}= k\!_\perp r\!_\perp \cos \Delta\phi +k_{z}{z}$, where $ \Delta\phi=\phi_{k}-\phi_{r} \in [0,2\pi)$ is the angle between $ \mathbf{e}_{k\!_\perp} $ and $ \mathbf{e}_{r\!_\perp} $. Thus
\begin{align}
e^{i\mathbf{k}\cdot \mathbf{r}}=e^{ik\!_\perp r\!_\perp\cos\Delta\phi}e^{ik_{z}{z}}
\end{align}
is a periodic function of $ \Delta\phi $ and hence can be expanded into a Fourier series given below. 
\begin{align}
e^{i\mathbf{k}\cdot \mathbf{r}} &= \sum_{m=-\infty}^{\infty}c_{m}(k\!_\perp r\!_\perp)e^{im\Delta\phi}e^{ik_{z}{z}},
\end{align}
where the coefficients $ c_{m}(k\!_\perp r\!_\perp) $ is associated with Bessel's first integral~\footnote{see Jackson's E\&M, p140, or \url{http://physics.stackexchange.com/questions/44761/plane-wave-expansion-in-cylindrical-coordinates}.}
\begin{align}
c_{m}(k\!_\perp r\!_\perp)&= \frac{1}{2\pi} \int_0^{2\pi} e^{ik\!_\perp r\!_\perp\cos \Delta\phi}e^{-im\Delta\phi}\mathrm{d}\Delta\phi\\
&=i^mJ_m(k\!_\perp r\!_\perp).
\end{align}
Therefore, we have
\begin{align}
e^{i\mathbf{k}\cdot \mathbf{r}} &=\sum_{m=-\infty}^{\infty}i^me^{im\Delta\phi}e^{ik_{z}{z}}J_m(k\!_\perp r\!_\perp).
\end{align}

Notice that the first kind of Bessel's function usually represent standing radial waves, while Hankel functions usually describe travelling waves. Using the relationships that $ J_{m}(k\!_\perp r\!_\perp)=H_{m}^{(1)}(k\!_\perp r\!_\perp)+H_{m}^{(2)}(k\!_\perp r\!_\perp) $, we can reexpress the plane wave in terms of incoming (at negative r) and outgoing (at positive r) as below.
\begin{align}
e^{i\mathbf{k}\cdot \mathbf{r}} &=\sum_{m=-\infty}^{\infty}i^me^{im\Delta\phi}e^{ik_{z}{z}} H_{m}^{(1)}(k\!_\perp r\!_\perp)+\sum_{m=-\infty}^{\infty}i^me^{im\Delta\phi}e^{ik_{z}{z}} H_{m}^{(2)}(k\!_\perp r\!_\perp).
\end{align}

This result shows that only when $ k_z=\beta $ can the tilt plane wave contributes to a nanofiber mode with wavenumber $ \beta $, since both $ k_z $ and $ \beta $ have consistent physics meaning. 