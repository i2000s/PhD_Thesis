\documentclass[fleqn, final]{../styles/unmeethesis}
\usepackage{../styles/qxd}
\renewcommand{\thechapter}{1}
%\newcommand{\thechapter}{1}

\makeindex
\begin{document}
%<*tag>

\renewcommand{\lb}[1]{\label{introduction:#1}}% creates chapter specific labels
\renewcommand{\rf}[1]{\ref{introduction:#1}}% same as before
%\newcommand{\lb}[1]{\label{introduction:#1}}% creates chapter specific labels
%\newcommand{\rf}[1]{\ref{introduction:#1}}% same as before

\chapter{Introduction}
\label{ch:introduction}

%\begin{quote}
%Today, if you have a demanding job for light, you use an optical laser. In the future, if there is a demanding job for atoms, you may be able to use an atom laser.\\[4pt]
%--  Wolfgang Ketterle\ai{Ketterle, Wolfgang}
%\end{quote}
%\url{http://cua.mit.edu/ketterle_group/Projects_1997/atomlaser_97/atomlaser_comm.html}

\noindent This is left for writing a brief introduction.


\section{Atom-light interfaces}

A review.

%\begin{figure}[ht] % This is generated by inkscape with the option ``PDF+Latex''. The PDF image is in the sub-folder ``images''. The path of the sub-folder is included by adding ``\graphicspath{{images/}}'' to zj.sty.
%   \centering
%   \def\svgwidth{0.85\textwidth} % set the image width, this is optional
%   \input{../Inputs/temperature_gradient.pdf_tex}
%   \caption[Temperature Gradient from a BEC to the Center of the Sun]{Temperature gradient from a BEC to the center of the sun.}
%   \label{fig:temperature_gradient}
%\end{figure}


\section{The philosophy of cooperativity}

For single atom and many.

%\begin{figure}[ht] %This figure is drew by Inkscape using the very handy tool 'parametric curves'. The fringe is the hard part, and I don't know how to do it elegantly (I just use a combination of 'cheating' methods to generate it). However, there is one command called '\pgfdeclarefunctionalshading' in the tikZ package which could be helpful. But, unfortunately, it requires knowing how to write PostScript codes.
%   \centering
%   \def\svgwidth{0.85\textwidth}
%   \input{../Inputs/double_well_interfer.pdf_tex}
%   \caption[A Double-Well Atom Interferometer]{A double-well atom interferometer~\cite{shin_atom_2004}: (i)~cool the atoms trapped in a single-well potential to form a BEC; (ii)~split the condensate by slowly deforming the single-well potential to a double-well potential; (iii)~apply ac Stark shift potentials to either of the two separated condensates; (iv)~turn off the double-well trapping potential, and let the condensates ballistically expand, overlap, and interfere; and (v)~take an absorption image.}
%   \label{fig:double_well_interfer}
%\end{figure}

\section{Manipulating atoms for the unknown}



\section{Outline of This Dissertation}



\section{Other Work}



%</tag>
%###################################################################################
\bibliographystyle{../styles/abbrv-alpha-letters-links}
\bibliography{../refs/Archive,../chap4/Nanofiber}
%%%%%%%%%%%%%%%%%%%%%%%%%%%%%%%%%%%%%%%%%%%%%%%%%%%%%%%%%%%%%%%%%%%%%%%%%%%%%%%%%%%%%

\printindex
%\cleardoublepage
%\thispagestyle{plain}
%\phantomsection
%\printindex{ai}{Author Index}
%\chaptermark{Author Index}
%\thispagestyle{plain}
%\printindex{si}{Subject Index}
%\chaptermark{Subject Index}
\end{document}
