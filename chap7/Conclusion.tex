\documentclass[fleqn, final]{../styles/unmphythesis}
\usepackage{../styles/qxd}
\renewcommand{\thechapter}{7}
%\newcommand{\thechapter}{1}

\makeindex
\begin{document}
%<*conclusion>

\renewcommand{\lb}[1]{\label{conclusion:#1}}% creates chapter specific labels
\renewcommand{\rf}[1]{\ref{conclusion:#1}}% same as before
%\newcommand{\lb}[1]{\label{introduction:#1}}% creates chapter specific labels
%\newcommand{\rf}[1]{\ref{introduction:#1}}% same as before

\chapter{Conclusion and Outlook}
\label{ch:conclusion}

\section{Summary of conclusions}
Atom-light interaction is a key topic to study in modern physics. A quantum interface to provide strong coupling between atoms and photons is critical for applications in quantum information processing and precise sensing. 
On the other hand, we also want weak interactions of atom to the environment so that the decoherence of atoms is much weaker than the useful coupling of light to the atoms.
As a quantitative concept to characterize atom-light coupling, we generalize ``cooperativity" to atom-waveguide quantum interfaces in this study in the context of QND measurement, in which the cooperativity is defined as the ratio between the quantum measurement strength and the characteristic decoherence rate. 
We have shown that the effective mode area can be written as two parts: one is the area for effective interaction of quantum measurement, and the other is the mode area of the input mode, which associates with the decoherence. 
This realization of the separating of mode areas allows us to enhance the QND measurement effect and measurement-induced squeezing effect by optimizing the geometry of the system, which includes the boundary shape of the waveguide, the relative position of atoms to the waveguide and the polarization axis of the input light, the internal state of the atoms, and the quantization axis that defines the atomic states. 
Quantitatively studying these unique features of nanophotonic waveguide interfaces that does not exist in the traditional free-space atom-light interface becomes the main theme for this dissertation.


This dissertation focuses on the theoretical aspects of the nanophotonic waveguide interface with trapped alkali atoms sitting in the waveguide's evanescent field. 
We consider atoms are trapped as an optical lattice near a waveguide and coupling to orthogonal fundamental guided modes of the waveguide.
The rotation angle of the Stokes vector of the guided light on the \Poincare sphere is used as the signal for us to detect the state of atoms, and to produce squeezed states.
Particularly, we take two waveguide geometries as examples--a cylindrical single-mode optical nanofiber and a square waveguide. 
However, the dispersive atom-light interaction theory we developed is written in a general language of Green's function tensors, which we have provided two different ways to calculate for arbitrary nanophotonic waveguides.  
Using the Green's tensors, we show the atom-light coupling is magnitudes stronger compared to the conventional free-space Gaussian beam trapping scheme.
We have also developed a theory based on the Green's tensors and modal decompositions to calculate the modified spontaneous emission rates of a multiple-level atom close by, and decompose the decay rates into guided and unguided mode components.
We suggested a range of trapping distances of atoms, above which we can surely ignore the modification of the decay rates due to the waveguide. 
Next, we derived an input-output theory for the light's dispersive response due to trapped atoms in the Heisenberg-Langevin picture, which reveals the phase shift of a guided mode after the interaction can be determined by the decay rate of the atoms coupled to the guided mode.
Therefore, with the mathematical model established, we can conclude that the light response to atoms is fully determined by the geometric properties of the waveguide, the position of the atoms, and the state of atoms; and we can enhance the atom-light coupling by controlling the factors above, which determines the coupling of atomic radiation to the guided modes. 
We have also provided a geometric explanation of radiation problems in presence of a dielectric interface for insights. 

To simulate the collective spin dynamics in the quantum measurement process, a set of stochastic master equations have also been developed for the atom-waveguide interfaces. 
The collective spin dynamics can be tracked by using a few approximations: first, we limit our calculation in a small subspace over the entire atomic hyperfine manifold; second, we assume there is a exchange symmetry over all atoms; third, we use Gaussian-state approximation to truncate high-order moments into one- and and two-body moments. 
We solve the spin dynamics by evolving microscopic one-body operators and the two-body covariances, and then solve for the collective operators and covariances using the microscopic solutions. 
This approach not only allows us to calculate the squeezing parameter easily on a computer, but also reveals the quantum correlation nature of measurement backtions.

As applications of the framework of theory, we propose a few protocols for precise quantum nondemolition (QND) measurement, measurement-induced spin squeezing using the birefringence and the Faraday effects.
In the birefringence-effect--based QND measurement protocols, we predict the sensitivity of counting the number of trapped atoms can be on the scale of $ 10 $.
To generate measurement-induced spin squeezing using the birefringence effect, we find magic frequencies to cancel a noise injection term in the Hamiltonian, and have found the optimal choice of quantization axis to maximize the squeezing when the decoherence processes are fully included in our model.
We have shown that $ \sim 4.7 $ dB of squeezing is achievable with $ 2500 $ atoms on a typical nanofiber interface in Chapter 5.
The negative correlations of pairwise atoms have also been witnessed in the process of generating spin squeezing. The squeezing reaches its peak value when the pairwise correlations decline to the minimum value. 
We have also shown the choice of quantization axis is not only determined by the birefringence interaction, but only relies on the decoherence rate. Therefore, we can validate our definition of the OD per atom or the cooperativity per atom quantity in the birefringence measurement case.

By defining and optimizing the cooperativity, we have designed a QND measurement and spin squeezing protocol based on the Faraday interaction for atoms prepared in the collective spin coherent state in the stretched state space.
We have shown in Chapter 6 that the squeezing can be improved to about $ 6 $ dB for the nanofiber case using similar configuration parameters as the birefringence case. 
When we consider a \SWG instead, the peak spin squeezing can be further improved to $ \sim 13 $ dB. 
Counter-intuitively, we found the optimal azimuthal positions to trap atoms in order to enhance the squeezing effect is where the local electric field is the weakest. 
We explained this result by checking the landscape of cooperativity.
This could potentially result in solving the dilemma that damaging effects increase when the atom-light coupling is enhanced according to the traditional way of thinking. 

In sum, we propose the atom-waveguide system as a promising quantum interface for implementing quantum data bus and quantum information processing applications. 
We have provided a dispersive light response theory to describe the atom-light interaction on this platform, and have defined the cooperativity formulas for QND measurement and spin squeezing protocols using a nanofiber and a \SWG, which may be used as a guideline for enhancing effective atom-light couplings. We have also revealed some new physics on these waveguide quantum interfaces over free-space platforms. 
Our theory could be used for generating non-Gaussian states, designing high-precision atomic clocks, and for other applications using atoms and light.


\section{Outlook}
Given the limited time and energy, I have to focus on a small fraction of areas working on for my PhD study. 
There are things I think they are important, and may be I will or someone else would like to continue for future research projects.
I list a few below for your reference.

So far, in calculating the collective spin dynamics, we haven't fully incorporated the modification of decay rates in our model. 
We have neither provided a solution of canceling the tensor light shift on atoms when the Faraday-interaction-based spin squeezing is calculated, although we assume there is a way to achieve it. 
These are the factors we have to consider if we want to further enhance the atom-light coupling by placing atoms much closer to the surface of the waveguide or by using an extra cavity at the two ends of the waveguides. 
One of the challenges to incorporate these two factors is that the tensor light shift of atoms, by its definition, could depend on both the internal state structure of the atoms and the modification of decay rates by the waveguide nearby the atoms. 
A two-color probe scheme has been developed for the free-space interface case~\cite{Montano2015Quantum}, where the modification of decay rates is not a problem as its state-independent. 
I have build a rough model to find a magic frequency canceling the internal tensor light shift of the atoms when the tensor effect of the modification of decay rates can be ignored. 
But to fully solve this problem, we may need to consider the irreducible tensor representation of the dyadic Green's function of the waveguide for different quantum transitions. 

Along the same line, when the coupling strength is enhanced high enough, the squeezed state could be non-Gaussian and our current model will no long hold.
Therefore, we eventually need to develop a quantum measurement theory that does not need to make the Gaussian state assumption and can cut-off many-body terms to a higher-order than two-body.
I think my current approach of using the microscopic operators and moments to evolve the spin dynamics is a valid way to begin with. 
Higher order moments should be easily kept in my formalism, and worth testing.

Another direction we were always speculating on is the state preparation and atomic cooling protocols on the atom-waveguide interfaces, but we haven't found a chance to solve it. 
There have been experimental reports on cooling the atoms to the mechanical ground state and preparing the hyperfine structure state of the trapped atoms using the resolved sideband cooling technique first developed for ion traps~\cite{Meng2017ground,Beguin2017Observation}. 
We think we can build a theory incorporate the full waveguide modification effects for resolved sideband cooling process using both guided and external lights, and gain some insights on the physics.
On the other hand, these techniques relies on pure $ \sigma_\pm $ and $ \pi $ transitions, but the experiments were done with an external light beam, which may not be the pure polarization states one needs due to surface scattering and imperfections. 
I think there may be a way to reuse the idea of cooperativity in the cooling and control case, which may be redefined as the ratio between the useful coupling rate over the population leaking and dephasing rates.
Then use a Floquet Hamiltonian for geometry optimizations. The fidelity of state preparation could be limited by the time of optical pumping rather than mainly by the imperfection of the local field.

Last but not least, since we already find some protocols of state readout and preparations, an immediate step next is to control the atoms or photons as a many-body system for quantum simulations, quantum memory and computing. 
One question is can we make the control of this system universal? 
Anyway, there is a rich repository of topics we can delve on top of what we have learned for better using the atom-waveguide interfaces.  




%</conclusion>
%###################################################################################
\bibliographystyle{../styles/abbrv-alpha-letters-links}
\bibliography{../refs/Archive,../chap5/Nanofiber}
%%%%%%%%%%%%%%%%%%%%%%%%%%%%%%%%%%%%%%%%%%%%%%%%%%%%%%%%%%%%%%%%%%%%%%%%%%%%%%%%%%%%%

\printindex
%\cleardoublepage
%\thispagestyle{plain}
%\phantomsection
%\printindex{ai}{Author Index}
%\chaptermark{Author Index}
%\thispagestyle{plain}
%\printindex{si}{Subject Index}
%\chaptermark{Subject Index}
\end{document}
