\documentclass[fleqn, final]{../styles/unmphythesis}
\usepackage{../styles/qxd}
\renewcommand{\thechapter}{7}
%\newcommand{\thechapter}{1}

\makeindex
\begin{document}
%<*conclusion>

\renewcommand{\lb}[1]{\label{conclusion:#1}}% creates chapter specific labels
\renewcommand{\rf}[1]{\ref{conclusion:#1}}% same as before
%\newcommand{\lb}[1]{\label{introduction:#1}}% creates chapter specific labels
%\newcommand{\rf}[1]{\ref{introduction:#1}}% same as before

\chapter{Conclusion and Outlook}
\label{ch:conclusion}

\section{Summary}
The atom-light interaction is a key topic to study in modern physics. A quantum interface to provide strong coupling between atoms and photons is critical for applications in quantum information processing and precise sensing. 
To achieve this, we also want weak interactions of atoms with the environment so that the decoherence of atoms is much weaker than the useful coupling of light to the atoms.
As a measure to characterize atom-light coupling, we generalize ``cooperativity" to the application of atom-waveguide quantum interfaces in this study in the context of QND measurement. 
We define the cooperativity as the ratio between the quantum measurement strength and the characteristic decoherence rate. 
These rates can, in tern, be defined in terms of characteristic areas.
We have shown that the effective mode area can be written as two parts: one is the area for effective interaction of quantum measurement, and the other is the mode area of the input mode, which is associated with the decoherence. 
This realization of the separating of mode areas allows us to enhance the QND measurement effect and measurement-induced squeezing effect by optimizing the geometry of the system, which includes the boundary shape of the waveguide, the relative position of atoms to the waveguide and the polarization axis of the input light, the internal state of the atoms, and the quantization axis that defines the atomic states. 
Our study of these unique features of nanophotonic waveguide interfaces that do not exist in the traditional free-space atom-light interface was the main theme for this dissertation.


This dissertation focuses on the theoretical aspects of the nanophotonic waveguide interface with alkali atoms trapped in the waveguide's evanescent field. 
We consider atoms that are trapped as an optical lattice near a waveguide and coupling to the two orthogonal polarizations of the fundamental guided modes of the waveguide.
The rotation angle of the Stokes vector of the guided light on the \Poincare sphere is used as the signal for us to detect the spin state of atoms, and to produce spin squeezed states.
Particularly, we take two waveguide geometries as examples--a cylindrical single-mode optical nanofiber and a square waveguide. 

More generally, the dispersive atom-light interaction theory we developed is written in a language of Green's function tensors, and we have provided two different ways to calculating this for arbitrary nanophotonic waveguides.  
Using the Green's tensors, we have shown that the atom-light coupling is orders of magnitudes stronger compared to the conventional free-space Gaussian beam trapping scheme.
We have also developed a theory based on the Green's tensors and modal decompositions to calculate the Purcell effect, i.e., the modified spontaneous emission rates of a multiple-level atom close by. We have decomposed the decay rates into guided and unguided mode components.
We suggested a range of trapping distances of atoms, above which we can surely ignore the modification of the decay rates due to the waveguide. 

We derived an input-output theory for the light's dispersive response due to trapped atoms in the Heisenberg-Langevin picture, which reveals the phase shift of a guided mode after the interaction can be determined by the emission rate of the atoms coupled to the guided mode.
Therefore, with the mathematical model established, we can conclude that the light response to atoms is fully determined by the geometric properties of the waveguide, the position of the atoms, and the state of atoms, and we can enhance the atom-light coupling by controlling the factors above, which determines the coupling of atomic radiation to the guided modes. 
We have also provided a geometric explanation of radiation problems in presence of a dielectric interface. 

To simulate the collective spin dynamics in the quantum measurement process, we developed a set of stochastic master equations for the atom-waveguide interfaces. 
The collective spin dynamics can be tracked by using a few approximations. 
First, we limit our calculation to a small subspace over the entire atomic hyperfine manifold. 
Second, we assume there is a exchange symmetry over all atoms. 
Third, we use the Gaussian-state approximation to truncate high-order moments into one- and and two-body moments. 
We solve the spin dynamics by evolving microscopic one-body operators and the two-body covariances, and then solve for the collective operators and covariances using the microscopic solutions. 
This approach not only allows us to calculate the squeezing parameter easily on a computer, but also reveals the quantum correlation nature of measurement backtions.

In applications of the framework of this theory, we proposed a few protocols for precise quantum nondemolition (QND) measurement-induced spin squeezing using the birefringence and the Faraday effects.
In the birefringence-effect--based QND measurement protocols, we predict the sensitivity of counting the number of trapped atoms can be on the scale of $ 10 $ atoms.
To generate measurement-induced spin squeezing using the birefringence effect, we find magic frequencies to cancel a noise injection term in the Hamiltonian, and have found the optimal choice of quantization axis to maximize the squeezing when the decoherence processes are fully included in our model.
We have shown that $ \sim 4.7 $ dB of squeezing is achievable with $ 2500 $ atoms on a typical nanofiber interface in Chapter 5.
%The negative correlations of pairwise atoms have also been witnessed in the process of generating spin squeezing. 
The squeezing reaches its peak value when the pairwise correlations create the most ``negative" terms than reduce the effect of uncorrelated spin project noise. 
We have also shown the choice of quantization axis is not only determined by the birefringence interaction, but only relies on the decoherence rate. 
With this, we can validate our definition of the OD per atom or, equivalently, the cooperativity per atom in the birefringence measurement case.

By defining and optimizing the cooperativity, we have designed a QND measurement and spin squeezing protocol based on the Faraday interaction for atoms prepared in the collective spin coherent state in the stretched state space.
We have shown in Chapter 6 that the squeezing can be improved to about $ 6 $ dB for the nanofiber case using similar parameters as the birefringence case. 
When we consider a \SWG instead, the peak spin squeezing can be further improved to $ \sim 13 $ dB. 
Counter-intuitively, we found the optimal azimuthal positions to trap atoms in order to enhance the squeezing effect is where the local electric field is the weakest. 
We explained in terms of the effective homodyne measurement performed in polarization spectroscopy.  
The local oscillator is the probe, and the signal is the photons in the orthogonal mode.  
We enhance the cooperativity by placing the atoms at the position where the orthogonal (unoccupied) mode is largest.
This could provide new insights into how the atom-light coupling is enhanced that goes beyond the traditional way of thinking. 

In sum, we propose the atom-waveguide system as a promising quantum interface for implementing a quantum data bus for quantum information processing applications. 
We have provided a dispersive light response theory to describe the atom-light interaction on this platform, and have defined the cooperativity formulas for QND measurement and spin squeezing protocols using a nanofiber and a \SWG. 
We have also revealed some new physics on these waveguide quantum interfaces over free-space platforms. 
Our theory could be used for generating non-Gaussian states, designing high-precision atomic clocks or magnetometers, and for other applications using atoms and light.


\section{Outlook}
No research progress is ever complete. 
There are many remaining topics I think are important, and it maybe I will or someone else would like to continue for future research projects.
I list a few below for your reference.

So far, in calculating the collective spin dynamics, we haven't fully incorporated the modification of decay rates in our model. 
We have neither provided a solution of canceling the tensor light shift on atoms when the Faraday-interaction-based spin squeezing is calculated, although we assume there is a way to achieve it. 
These are the factors we have to consider if we want to further enhance the atom-light coupling by placing atoms much closer to the surface of the waveguide or by using an extra cavity at the two ends of the waveguides. 
One of the challenges to incorporate these two factors is that the tensor light shift of atoms, by its definition, could depend on both the internal state structure of the atoms and the modification of decay rates by the waveguide nearby the atoms. 
A two-color probe scheme has been developed for the free-space interface case~\cite{Montano2015Quantum}, where the modification of decay rates is not a problem as its state-independent. 
I have build a rough model to find a magic frequency canceling the internal tensor light shift of the atoms when the tensor effect of the modification of decay rates can be ignored. 
But to fully solve this problem, we may need to consider the irreducible tensor representation of the dyadic Green's function of the waveguide for different quantum transitions. 

Along the same line, when the coupling strength is sufficiently enhanced, the squeezed state could be non-Gaussian and our current model will no long hold.
Therefore, we eventually need to develop a quantum measurement theory that does not need to make the Gaussian state assumption and can cut-off many-body terms to a higher-order than two-body.
Our current approach of using the microscopic operators and moments to evolve the spin dynamics is a valid way to begin with. 
Higher order moments should be easily kept in my formalism.

Another direction we have not fully explored is the state preparation and atomic cooling protocols on the atom-waveguide interfaces. 
There have been experimental reports on cooling the atoms to the mechanical ground state and preparing the hyperfine structure state of the trapped atoms using the resolved sideband cooling technique first developed for ion traps~\cite{Meng2017ground,Beguin2017Observation}. 
We can build a theory incorporate the full waveguide modification effects for resolved sideband cooling process using both guided and external lights, and gain some insights on the physics.
On the other hand, these techniques rely on pure $ \sigma_\pm $ and $ \pi $ transitions, but the experiments were done with an external light beam, which may not be the pure polarization states one requires due to surface scattering and other imperfections. 
There may be a way to reuse the idea of cooperativity in the cooling and control case, which may be redefined as the ratio between the useful coupling rate over the population leaking and dephasing rates.
By using a Floquet Hamiltonian, one can optimize the geometry. The fidelity of state preparation could be limited by the rate of optical pumping rather than mainly by the imperfection of the local field.

In the longer term, since given our protocols for state readout and preparations, an immediate step next step is to control the atoms or photons as a many-body system for quantum simulations, quantum memory and computing. 
One question is can we make the control of this system universal? 
There is a rich repository of topics we can explore on top of what we have learned for applications of the atom-waveguide interfaces.  




%</conclusion>
%###################################################################################
\bibliographystyle{../styles/abbrv-alpha-letters-links}
\bibliography{../refs/Archive,../chap5/Nanofiber}
%%%%%%%%%%%%%%%%%%%%%%%%%%%%%%%%%%%%%%%%%%%%%%%%%%%%%%%%%%%%%%%%%%%%%%%%%%%%%%%%%%%%%

\printindex
%\cleardoublepage
%\thispagestyle{plain}
%\phantomsection
%\printindex{ai}{Author Index}
%\chaptermark{Author Index}
%\thispagestyle{plain}
%\printindex{si}{Subject Index}
%\chaptermark{Subject Index}
\end{document}
