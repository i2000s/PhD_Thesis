%=========== APPENDIX ===========%
\chapter[Deriving birefringence QND measurement dynamics]{Some detailed derivation of the birefringence QND measurement dynamics}


%=========== APPENDIX: Nanofiber mode functions ===========%
\section{Guided-mode functions for the optical nanofiber} \label{Appendix::ModeFunctions}

  

In this Appendix we provide, for reference, the fundamental HE$_{11}$ solutions to the homogeneous wave equation, \erf{Eq::WaveEquationSource} with $\tensor{\boldsymbol{\alpha}} = 0$, for a cylindrical nanofiber of radius $a$ and index of refraction given by \erf{Eq::IndexofRefraction}.  At a given frequency, $\omega_0 = c k_0$, the magnitudes of the longitudinal and transverse wave vectors for a guided mode are related by $n^2 k_0^2 = \beta_0^2 + k_\perp^2$.  
The positive propagation constant, $\beta_0 \equiv \beta(\omega_0)$, is determined from the eigenvalue equation that results from enforcing physical boundary conditions at the fiber surface \cite{snyder_optical_1983},
	\begin{align}
		\frac{J_0(ha)}{ha J_1(ha)} = - \frac{n_1^2+n_2^2}{2n_1^2} \frac{K'(qa)}{qa K_1(qa)} + \frac{1}{h^2 a^2} - \bigg[ \bigg(\frac{n_1^2 - n_2^2}{2 n_1^2} \frac{K'(qa)}{qa K_1(qa)} \bigg)^2  + \frac{\beta_0^2}{n^2_1 k^2} \bigg(\frac{1}{q^2a^2} + \frac{1}{h^2a^2} \bigg)^2 \bigg]^{1/2}.
	\end{align}
Inside the nanofiber the transverse wavevector is real, $k_\perp = q$, where $q=\sqrt{\beta_0^2- n_2^2k_0^2}$, and outside the nanofiber it is purely imaginary, $k_\perp = i h$, where $h=\sqrt{n_1^2 k_0^2 - \beta_0^2}$.  The vector eigenfunctions are expressed as $\mbf{f}_{\mu}(\br) = (2\pi)^{-1/2}\mbf{u}_{b,p}(\mbf{r}_\perp) e^{i b \beta_0 z}$, where the modes are indexed by frequency $\omega_0$, propagation direction $b = \pm$, and polarization $p$.

A relatively simple form for the guided-mode functions can be expressed in a cylindrical basis $(r_\perp, \phi, z)$ with longitudinal unit vector $\mathbf{e}_z$, oriented along the fiber axis.  
The transverse unit vectors are related to their fixed Cartesian counterparts via the relations
\begin{subequations}
	\begin{align}
		\mathbf{e}_{r_{\!\perp}}     &= \mathbf{e}_x \cos \phi + \mathbf{e}_y \sin \phi, \\
		\mathbf{e}_\phi &= - \mathbf{e}_x \sin \phi + \mathbf{e}_y \cos \phi.
	\end{align}
\end{subequations}
The transverse profile for the quasicircular guided modes, $p = \pm$, is
	\begin{align} \label{Eq::QuasicircularModes}
		\mbf{u}_{b,\pm}(\mathbf{r}_\perp) = \big[\mathbf{e}_{r_{\!\perp}} u_{r_{\!\perp}}(r_\perp) \pm i \mathbf{e}_\phi u_\phi(r_\perp) +  i b \mathbf{e}_z  u_z(r_\perp) \big]e^{ \pm i \phi}, 
	\end{align}
and for the quasilinear guided modes, $p = \{H,V\}$, is
	\begin{subequations} \label{Eq::QuasilinearModes}
	\begin{align}
		\mbf{u}_{b,H}(\mathbf{r}_\perp) = & \sqrt{2} \big[ \mathbf{e}_{r_{\!\perp}} u_{r_{\!\perp}}(r_\perp) \cos \phi - \mathbf{e}_\phi u_\phi(r_\perp) \sin \phi +  ib \mathbf{e}_z  u_z(r_\perp) \cos \phi \big] \\
		\mbf{u}_{b,V}(\mathbf{r}_\perp) = & \sqrt{2} \big[ \mathbf{e}_{r_{\!\perp}} u_{r_{\!\perp}}(r_\perp) \sin \phi + \mathbf{e}_\phi u_\phi(r_\perp) \cos \phi +  ib \mathbf{e}_z  u_z(r_\perp) \sin \phi \big]. 
	\end{align}
	\end{subequations}
The modes are expressed in terms of real-valued functions that depend only on the radial coordinate $r_\perp$,
	\begin{subequations} \label{Eq::ProfileFunctions}
	\begin{align} 
		u_{r_{\!\perp}}(r_\perp) =& u_0 \big[ (1-s) K_0(q{r_{\!\perp}}) + (1+s)K_2(q{r_{\!\perp}})\big] \\
		u_\phi(r_\perp) =& u_0\big[ (1-s) K_0(q{r_{\!\perp}}) - (1+s)K_2(q{r_{\!\perp}})\big] \\
		u_z(r_\perp) =& u_0 \frac{2 q}{\beta_0} \frac{K_1(qa)}{J_1(ha)} J_1(h{r_{\!\perp}}), \label{Eq::zprofile}
	\end{align}
	\end{subequations}
where $u_0$ is set by the normalization condition, $\int d^2 \mathbf{r}_\perp n(r_\perp) | \mathbf{u}_\mu(\br_\perp)|^2=1$, $J_n$ and $K_n$ are the $n^{th}$ Bessel functions of the first and second kind, $f'(x)$ indicates a derivative with respect to the argument $x$, and 
	\begin{align}
		s = \frac{1/(q^2 a^2)^{2} + 1/(h^2 a^2)^{2}}{[J'_1(ha)/haJ_1(ha) + K'_1(qa)/qaK_1(qa)]}.
	\end{align}  
Of particular interest is the $z$-component, \erf{Eq::zprofile}, which can become appreciable.  Note that the phase convention in Eqs. (\ref{Eq::QuasicircularModes}-\ref{Eq::ProfileFunctions}) has been chosen to emphasize properties of the quasilinear modes and differs from that of \emph{Le Kien et al.} -- for instance in Ref. \cite{le_kien_propagation_2014}.  
Further details about the guided-mode fields inside the nanofiber ($r_\perp\leq a$), the radiation (unguided) modes, and the quantized form of both can be found in Refs. \cite{sondergaard_general_2001, tong_single-mode_2004, kien_field_2004, le_kien_spontaneous_2005, vetsch_eugen_optical_2010}.


%===================APPENDIX: Photon scattering and optical pumping rates =====================%
\section{Photon scattering and optical pumping rates} \label{Appendix::Rates}	

In this Appendix we give the explicit expressions for the photon scattering rates used in Sec.~\ref{Sec::QNDMeasurement} following the formalism given in~\cite{deutsch_quantum_2010}.  The total rate of photon scattering by an atom in the clock state $\ket{f,0}$ is
	\begin{equation}\label{Eq::gammaf}
		\gamma_{f}=- \frac{2}{\hbar} {\rm Im} \big[ \bra{f,0} \hat{h}_{\rm eff}\ket{f,0} \big] ,
	\end{equation}
where the effective non-Hermitian light-shift Hamiltonian for one atom is 
\begin{align}
\hat{h}_{\rm eff} = - \hat{\mathbf{E}}^{(-)}_{\rm in}(\mathbf{r}' ; t ) \cdot \poltens \cdot \hat{\mathbf{E}}^{(+)}_{\rm in}(\mathbf{r}' ;t )
\end{align}
as follows from \erf{Eq::LightShiftHam}, where $\charpol = -\frac{\sigma_0}{8\pi k_0}\frac{\Gamma}{\Delta_{ff'}+i\Gamma/2}$ is the complex polarizability and the irreducible tensor operator $ \hat{\tensor{\mbf{A}}}(f,f') $ is given in \erf{Eq::PolarizabilityIrrep}.


The rate of optical pumping between clock states $\ket{f,0} \rightarrow \ket{\tilde{f},0}$ is
	\begin{equation}\label{Eq::gammaff}
		\gamma_{f \rightarrow \tilde{f} } 
		=\sum_{q}\big| \bra{\tilde{f},0} \hat{W}_q^{\tilde{f}f} \ket{f,0} \big|^2,
	\end{equation}
where $ \hat{W}_q^{\tilde{f}f} = \sum_{f'}\frac{\Omega/2}{\Delta_{f'\tilde{f}}+i\Gamma/2}(\mathbf{e}_q^*\cdot\hat{\mathbf{D}}_{\tilde{f} f'} )(\mathbf{e}_{\rm in}\cdot \hat{\mathbf{D}}^\dagger_{f'f} ) $ are the Lindblad jump operators for optical pumping between ground levels $ f\rightarrow \tilde{f} $~\cite{deutsch_quantum_2010}. 
Each jump operator $\hat{W}_q^{\tilde{f}f}$ is associated with absorption of the probe photon polarized along $ \mathbf{e}_{\rm in} $ followed by spontaneous emission of a photon with polarization $ \mathbf{e}_q $, where $q= \{0,\pm 1\}$ labels spherical basis elements for $\pi$ and $ \sigma_\pm$ transitions.  

To find the dependence on the input field intensity, we define a characteristic photon scattering rate, $\gamma_s \equiv \frac{\Gamma\Omega^2}{4\Delta_{J_3}^2}= \frac{\sigma_0}{A_{\rm in}}\frac{\Gamma^2}{4 \Delta_{J_3}^2} \dot{N}_L $, with Rabi frequency $ \Omega=2\bra{j}|d|\ket{j'}\mathcal{E}^{(+)}_{\rm in}/\hbar $, reduced optical dipole matrix element $\bra{j}|d|\ket{j'}$, and field amplitude $ \mathcal{E}^{(+)}_{\rm in}=|\mathbf{E}_{\rm in}^{(+)}(\br')| $.
Eqs.~\eqref{Eq::gammaf} and~\eqref{Eq::gammaff} yield,
\begin{subequations}
	\begin{align}
		\gamma_f &=n_g\dot{N}_L  \sum_{f'} \sigma (\Delta_{ff'} ) \mathbf{u}^*_\inp(\br'_\perp)\cdot \bra{f,0} \hat{\tensor{\mbf{A}}}(f,f') \ket{f,0}  \cdot \mathbf{u}_\inp(\br'_\perp)\\
		&\approx  \gamma_s \sum_{f'} \frac{\Delta_{J_3}^2}{\Delta_{ff'}^2}\sum_q \big| o_{jf}^{j'f'}C_{f'q}^{f0;1 q} \big|^2 \mathbf{e}_q^* \cdot (\mathbf{e}_{\rm in}\mathbf{e}_{\rm in}^* )\cdot \mathbf{e}_q, 
	\end{align}
\end{subequations}
	\begin{align}
		\gamma_{f \rightarrow \tilde{f}} 
		&\approx \gamma_s \sum_{f'} \frac{\Delta_{J_3}^2}{\Delta_{ff'}^2}\sum_q \big| o_{j\tilde{f}}^{j'f'} o_{jf}^{j'f'}C_{f'q}^{\tilde{f}0;1 q}C_{f'q}^{f0;1q} \big|^2 \mathbf{e}_q^* \cdot (\mathbf{e}_{\rm in}\mathbf{e}_{\rm in}^* )\cdot \mathbf{e}_q,
	\end{align}
where $ \sigma (\Delta_{ff'} )  = \sigma_0 \Gamma^2/4\Delta^2_{f' f}$ is the the scattering cross section at the probe detuning, $ C_{f'q}^{f0;1 q}=\Braket{f'q}{f0;1q}$ are the Clebsch-Gordan coefficients, and
\begin{equation}
\big| o_{jf}^{j'f'} \big|^2=(2j'+1)(2f+2) \bigg\{
\begin{array}{ccc}
f' & 7/2 & j' \\
 j & 1 & f 
 \end{array}
 \bigg\}
\end{equation}
are the relative oscillator strengths determined by the relevant Wigner 6-$J$ symbol.

%===================APPENDIX: Equations of motion =====================%
\section[Equations of motion for the moments]{Derivation of the equations of motion for the moments} \label{Appendix::OpticalPumping}	

In this Appendix we derive the equations of motion for the correlation functions that define the metrologically relevant squeezing parameter, $\xi^2 = N_A \Delta J_3^2/\expt{\hat{J}_1}^2$.  
We seek the time evolution of the one and two-body correlation functions:
\begin{subequations}
\begin{align}
&\expt{\hat{N}_C} = \sum_n \expt{\hat{\mathbbm{1}}^{(n)}_C} \\
&\expt{\hat{J}_1} = \frac{1}{2} \sum_n \expt{\hat{\sigma}_1^{(n)}} \\
&\expt{\hat{J}_3} = \frac{1}{2} \sum_n \expt{\hat{\sigma}_3^{(n)}} \\
&\expt{\hat{J}_3^2} = \frac{\expt{\hat{N}_C}}{4} +\frac{1}{4} \sum_{m \neq n} \expt{\hat{\sigma}_3^{(m)}\otimes \hat{\sigma}_3^{(n)}}, 
\end{align}
\end{subequations}
where $\hat{\mathbbm{1}}_C \equiv \op{\uparrow}{\uparrow} + \op{\downarrow}{\downarrow}$ is the single-atom projector onto the clock states. 
To include optical pumping, we apply the following equations of motion. For a collective, single-body operator, $\hat{X} = \sum_n \hat{x}^{(n)}$, the evolution due to optical pumping is $d\expt{ \hat{X}}|_{\rm op} = \sum_n \Tr [\mathcal{D}_n[\hat{\rho}]\hat{X} ]dt = \sum_n \expt{\mathcal{D}_n\dg[\hat{x}^{(n)}]} dt$, where the map, which acts locally on atoms along the nanofiber, is given in \erf{Eq::OpticalPumpingMapSchr}.  
Two-body microscopic operators decay by optical pumping according to \cite{baragiola_three-dimensional_2014}
	\begin{align} \label{Eq::TwoBodyDecay}
		\frac{d}{dt} \expt{ \hat{x}^{(m)} \otimes \hat{y}^{(n)} } \Big|_{\rm op}= &\expt{ \mathcal{D}_m\dg[ \hat{x}^{(m)}] \otimes \hat{y}^{(n)} } + \expt{ \hat{x}^{(m)}\otimes \mathcal{D}_n\dg[ \hat{y}^{(n)}] },
	\end{align}
where the superscripts refer to the $m^{th}$ and $n^{th}$ atoms. 

Applying the adjoint map to the single-atom operators yields 
	\begin{subequations} \label{Eq::OperatorMap}
	\begin{align}
		\mathcal{D}\dg[\hat{\mathbbm{1}}_C] & = - \gamma_{00} \hat{\mathbbm{1}}_C +\gamma_{03} \hat{\sigma}_3 \label {Eq::Idecay} \\
		\mathcal{D}\dg[\hat{\sigma}_3] & =- \gamma_{33} \hat{\sigma}_3 +  \gamma_{30} \hat{\mathbbm{1}}_C 
\label{Eq::zdecay} \\
		\mathcal{D}\dg[\hat{\sigma}_1] & = - \gamma_{11} \hat{\sigma}_1\label{Eq::xdecay},
	\end{align}
	\end{subequations}
with rates	
	\begin{subequations} \label{Eq::DecayRates}
	\begin{align}
		\gamma_{00} 
			& = \frac{\gamma_{\uparrow}+\gamma_{\downarrow} - \gammauu-\gammaud  -\gammadd-\gammadu}{2} \label{Eq::lrate} \\
			\gamma_{03} 
			& = \frac{-\gamma_{\uparrow}+\gamma_{\downarrow} +\gammauu + \gammaud - \gammadd - \gammadu }{2}\\		
		\gamma_{33} 
			& = \frac{\gamma_{\uparrow}+\gamma_{\downarrow} - \gammauu+\gammaud  -\gammadd+\gammadu}{2}\\
			\gamma_{30} 
			& = \frac{-\gamma_{\uparrow} + \gamma_{\downarrow} + \gammauu - \gammaud - \gammadd + \gammadu }{2} \\
			\gamma_{11} 
			& = \frac{\gamma_{\uparrow}+\gamma_{\downarrow}}{2}. \label{Eq::frate}
	\end{align}
	\end{subequations}
Given Eqs. (\ref{Eq::TwoBodyDecay}, \ref{Eq::OperatorMap}), the equations for the two-body spin correlations, \erf{Eq::TwoBodySpinDecay}, follow.  Similarly, one can derive equations of motion for the remaining two-body microscopic operator correlations $ \expt{\hat{\mathbbm{1}}^{(m)}_C \otimes \hat{\mathbbm{1}}^{(n)}_C} $ and $ \expt{\hat{\mathbbm{1}}^{(m)}_C \otimes \hat{\sigma}_3^{(n)} + \hat{\sigma}_3^{(m)} \otimes \hat{\mathbbm{1}}^{(n)}_C } $ when $ m\neq n $ and from these, the macroscopic operator expectation values $ \expt{\hat{J}_3^2} $, $ \expt{\hat{N}_C^2} $, and $ \expt{\hat{N}_C\hat{J}_3} $.  As we have examined numerically, on the time scale of the QND measurement, the correlation between atom number in the clock state subspace and the pseudospin moment is weak, and one can thus treat the atom number operator in the clock state subspace as a $c$-number.
We therefore set $ \expt{\hat{N}_C\hat{J}_3}-\expt{\hat{N}_C}\expt{\hat{J}_3} = 0 $ and $ \expt{\hat{N}_C^2} - \expt{\hat{N}_C}^2 = 0 $ and define $ N_C\equiv \expt{\hat{N}_C}$. 

The equations of motion for the moments of $\jz$ are now found from the SME, \erf{Eq::SME},
	\begin{subequations} \label{Eq::J3MomentEquations}
	\begin{align} 
		d \expt{\jz} =& s \sqrt{\kappa} \varz \, dW - \gamma_{33} \expt{\jz}dt + \smallfrac{1}{2} \gamma_{30} N_C dt ,  \\
		d \expt{\jz^2} =& 2 s\sqrt{\kappa} \expt{\jz}\Delta J_3^2 \, dW - 2 \gamma_{33} \expt{\jz^2}dt + \smallfrac{1}{4} \big( 2 \gamma_{33}-\gamma_{00}\big) N_C dt \\
		&+ \gamma_{30} \expt{\jz} N_C dt + \smallfrac{1}{2}\left(\gamma_{03} -2 \gamma_{30}\right) \expt{\jz} dt. \nonumber 
	\end{align}
	\end{subequations}
The stochastic term in $d\expt{\jz^2}$ was simplified by assuming Gaussian statistics \cite{jacobs_straightforward_2006}, $\expt{\jz^3} = 3\expt{\jz^2}\expt{\jz}- 2\expt{\jz}^3$. 
Finally, the It\={o} calculus governing the stochastically evolving moments requires that differentials be taken to second order, and the evolution of the variance is given by $d \varz = d \expt{\jz^2} - 2 \expt{\jz} d \expt{\jz} - ( d \expt{\jz} )^2$. 
The equation of motion for the conditional variance, \erf{Eq::varJz}, then follows from Eqs. \eqref{Eq::J3MomentEquations}.
