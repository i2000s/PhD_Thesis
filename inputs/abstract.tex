\begin{abstract}

\noindent 
A quantum interface to provide strong coupling between atoms and photons is critical to implement applications in quantum information processing and precise sensing that quantum mechanics promise. 
In addition, we require weak interaction of atom to the environment so that the decoherence of atoms is much weaker than the useful coupling of light to the atoms.
Atoms trapped near the surface of a nanophotonic waveguide is a promising platform for achieving this strong coupling regime.  
As a measure to atom-light coupling, we generalize cooperativity to the waveguide interface.
%case in this study to describe the ratio between useful coupling and the decoherence rate. 
%Despite the spontaneous emission of atoms coupled to the coupled mode could be weak (or the relative ratio of $ \Gamma_{1D} $ to the natural line width of atoms) for a waveguide in the dispersive regime, with a lot of atoms interacting with the light, the total cooperativity can be high enough to produce interesting phenomena.
\qxd{may not be useful here: Meanwhile, the dispersive interaction nature where atoms are only adiabatically coupled to the light ensures the photon scattering among atoms can be negligible that reduces the risk of kicking atoms away by the probe light from being trapped.
This scheme of quantum interface also makes all atoms as redundant copies so that quantum measurement becomes a good subject to study to demonstrate some unique features of quantum measurement and to generate non-classical collective states via quantum measurement due to the strong coupling between atom and light, which may be hardly attainable if the atom-light interaction is weak.}
This dissertation focuses on dispersive response of such a nanophotonic system as a mean to create an entangling atom-light interface.
We consider atoms are trapped as a two-chain optical lattice with defects on two sides of a waveguide, and are manipulated using the orthogonal fundamental guided modes of the probe light.
Particularly, I will use two waveguide geometries as examples--a cylindrical single-mode optical nanofiber and a square waveguide--and show the atom-light coupling is magnitudes stronger compared to the conventional free-space Gaussian beam trapping scheme.
Theories based on dyadic Green's function method will be derived to characterize the properties of the waveguides and the modified spontaneous emissions of the atoms close by. 
A set of master equations will be established to describe the input-output relations of the guided modes of a waveguide due to atom-light interaction in the dispersive regime. 
A set of stochastic master equations will also be established in studying the collective spin dynamics in presence of the nanophotonic interface.
As applications of the framework of theory, I propose a few protocols for precise quantum nondemolition (QND) measurement, measurement-induced spin squeezing based on the birefringence and Faraday effects, as well as some state preparation procedures.
The magic frequencies, the optimal choice of quantization axis and the geometry of probes will be discussed in details.
In studying specific protocols, I will also provide some physical insights on designing the geometry of probe schemes and choosing the quantization axis towards more general cases of quantum measurement and control tasks based on the definition of cooperativity and its optimization. 
These protocols cover the readin and readout procedures and form a foundation towards a fully controllable quantum data bus and quantum information processing interface via strong collective coupling of the light with neutral atoms. 

\qxd{Need to compress to less than 350 words.}
\end{abstract} 