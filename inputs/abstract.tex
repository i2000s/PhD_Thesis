\begin{abstract}

\noindent 
Strong coupling between atoms and light is critical for quantum information processing and precise sensing. 
A nanophotonic waveguide is a promising platform for realizing an atom-light interface that reaches strong coupling regime.  
In this dissertation, we study the dispersive response theory of the nanowaveguide system as a mean to create an entangling atom-light interface, with applications to quantum non-demolition (QND) measurement and spin squeezing. 

We calculate the dyadic Green's function, which determines the scattering of light by atoms in the presence of a nanowaveguide, and thus the phase shift and polarization rotation induced on the guided light. 
We solve for the Green's tensor for a cylindrical nanofiber and a square waveguide analytically and numerically. 
The Green's function is related to a full Heisenberg-Langevin treatment of the dispersive response of the quantized field to tensor polarizable atoms.
Using the Green's tensors, we calculate the mediated spontaneous emission rates for classical dipoles and alkali atoms. 

We model QND measurement and spin squeezing using a first-principles stochastic master equations. 
Based on the birefringence effect, we propose a spin squeezing protocol for spins encoded in the clock transition of cesium-133. 
We generalize the concept of cooperativity, which is determined by the ratio between the measurement strength and the decoherence rate in the context of the dispersive waveguide interface.
By maximizing the cooperativity per atom, we find the optimal choice of quantization axis that defines the clock states, and we predict a peak squeezing of $ 4.7 $ dB with $ 2500 $ atoms trapped along a realistic nanofiber.

To further enhance squeezingand for applications in magnetometry, we propose a protocol based on the Faraday effect for a nanofiber and a \SWG with magnetic spin stretched states. 
Counterintuitively, by placing the atoms at an azimuthal position where the guided probe mode has the lowest intensity, we increase the cooperativity.
This arises because the measurement strength depends on the interference between the probe and scattered light into an orthogonal mode, while the decoherence rate depends on the local intensity of the probe. Thus, by proper choice of geometry, the ratio of good-to-bad scattering can be strongly enhanced for highly anisotropic modes.
We find $ 6.3 $ dB and $ 13 $ dB of peak squeezing for the nanofiber and the \SWG, respectively, with $2500$ atoms.

%Our theory could be helpful for designing nanowaveguide-atom interfaces to enhance the atom-light coupling in applications to atomic clocks, atomic interferometers, quantum data bus and other quantum information devices. 

\end{abstract} 