\begin{abstract}

\noindent 
A quantum interface to provide strong coupling between atoms and light is critical to implement protocols in quantum information processing and precise sensing. 
Atoms trapped near the surface of a nanophotonic waveguide is a promising platform to reach strong coupling regime.  
In this dissertation, we focus on the theory to describe the dispersive response of the nanowaveguide system as a mean to create entangling atom-light interface, with applications to quantum non-demolition (QND) measurement and measurement-induced spin squeezing. 

We use the dyadic Green's function method to describe the properties of the waveguide, and solve the Green's tensor for a cylindrical optical nanofiber and a square waveguide analytically and numerically. An input-output relation of the guided modes of a waveguide is derived using Heisenberg-Langevin equations in the dispersive regime.

We develop a theory for the mediated spontaneous emission rates of classical dipoles and then generalize it to include the multiple-level alkali atoms. 
We use the Green's tensor to calculate the modified photon emission rates for a $ ^{133} $Cs atom near a nanofiber and a square waveguide. 

We derive a theory for QND measurement and spin squeezing using a set of stochastic master equations. 
Based on the birefringence effect, we propose a spin squeezing protocol on clock transitions, which predicts about $ 4.7 $ dB of squeezing may be achievable with $ 2500 $ atoms trapped along a nanofiber with realistic experimental parameters.

To further enhance the atom-light coupling and spin squeezing, we propose a protocol based on the Faraday effect for both a nanofiber and a \SWG with magnetic spin stretched states. 
We find about $ 6 $ dB and $ 13 $ dB of squeezing may be attainable for the nanofiber and the \SWG, respectively, with $ 2500 $ atoms and practical parameters.
We generalize the concept of cooperativity to the waveguide interface for insights of optimizing the waveguide and measurement geometry to enhance the squeezing effect.
We find the optimal azimuthal trapping position is counter-intuitively at the weakest field places. 

Our theory could be helpful for designing nanowaveguide-atom interfaces to enhance the atom-light coupling in applications to atomic clocks, atomic interferometers, quantum data bus and other quantum information devices. 

\end{abstract} 